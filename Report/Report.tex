\documentclass[11pt]{report}

\usepackage[utf8]{inputenc}
\usepackage{algorithm}
\usepackage{algpseudocode}
\usepackage{hyperref}
\usepackage{graphicx}
\usepackage{subcaption}
\usepackage{booktabs}
\usepackage[square,sort,comma,numbers]{natbib}
\bibliographystyle{abbrv}
\usepackage[font=small,labelfont=bf]{caption}

% Title Page
\title{\textbf{Sports Classification - Temporal vs. Static}}
\author{Object Recognition and Image Understanding \\ \\
  Project Report by \\
  Dominique Cheray and Manuel Krämer}

\begin{document}
\maketitle

\tableofcontents
 
\chapter{Introduction}

\chapter{Fundamentals}
\section{GoogLeNet}
\subsubsection{by Dominique Cheray}
GoogLeNet is a deep convolutional neural network first presented in 2014 and
winner of the ImageNet Large-Scale Visual Recognition Challenge of the same
year. The main building block of this network are the Inception modules which
reduce the number of parameters and at the same time create a deeper and wider
topology. \\
Increasing the size of a deep neural network is the most straightforward way to
improve its performance. This includes increasing the depth of the network, i.e.
the number of levels and the width of the network, i.e. the number of units at
each level. However this approach has two major drawbacks: a bigger size of the
network usually means more parameters, making the network more prone to
overfitting. Furthermore increasing the size of the network also drastically
increases the use of computational resources. To fundamentally solve these
problems, one would have to switch from fully connected to sparsely connected
architectures. However, today's computing infrastructures are very inefficient
when it comes to numerical calculation on non-uniform sparse data structures.
Therefore the main idea of the Inception modules is based on approximating and covering an
optimal local sparse structure in a convolutional network by readily available dense
components \cite{szegedy2015going}. So the authors aim to find the optimal local
construction and to repeat it spatially. \\
They argue that an optimal layered network topology can be constructed by analyzing the
correlation statistics of the preceding layers and clustering units with highly
correlated outputs. In lower layers correlations would concentrate in local and
near-local regions and can therefore be covered by 1x1 convolutions.
Additionally, a smaller number of spatially spread-out clusters can be covered
by convolution over larger patches, i.e. 3x3 an 5x5. The final architecture is a
combination of all those layers with their outputs concatenated into a single
vector which is then used as input to the next layer. Additionally an
alternative pooling path is added since pooling operations have been essential
for the success in convolutional networks \cite{szegedy2015going}. To avoid
computational blow up dimension reduction is applied wherever the computational
requirements would increase too much otherwise. Therefore inexpensive 1x1
convolutions are used to compute reductions before the expensive 3x3 and 5x5
convolutions. The 1x1 convolutions also include a ReLU activation making them
dual-purpose. Figure \ref{InceptionModule} shows the structure of an Inception
module. \\
For the final layout of the network several Inception modules are stacked
upon each other. To decimate the resolution of the grid max-pooling layers are
inserted occasionally.For reasons of memory efficiency lower layers are kept in
traditional convolutional fashion end the Inception modules are only used at
higher levels. This stacking allows for tweaking each module without
uncontrolled blow up in computational complexity. The network ends with global
average pooling followed by dropout and a fully connected layer with softmax for the
classification. The network is 27 layers deep and the overall number of layers
used for its construction is about 100.
\begin{figure}
  \centering
  \includegraphics[width=0.5\textwidth]{InceptionModule}
  \caption{Inception module with dimension reduction (taken from \cite{szegedy2015going}).}
  \label{InceptionModule}
\end{figure}

\section{ResNet}
\subsubsection{by Dominique Cheray}
Residual Networks were first presented in 2015 and winner of the ImageNet
Large-Scale Visual Recognition Challenge of the same year. The core idea of
ResNet is the introduction of so-called identity shortcut connections that skip
one or more layers, as shown in figure \ref{ResidualBlock}. A residual block has
two options: either perform a set of functions on the input or skip this step
altogether. \\
The depth of a neural network is of crucial importance for its performance.
However, increasing network depth does not work by simply
stacking layers together. With increasing network depth the accuracy gets saturated
and eventually degrades rapidly and therefore adding more layers to a suitably
deep model leads to higher training error \cite{he2016deep}.
The authors of \cite{he2016deep} argue that stacking layers shouldn't degrade the networks
performance, because one could simply stack identity mappings upon the current
network and the resulting architecture would perform the same. This indicates
that the deeper model should not produce a higher training error than its
shallower counterpart. But optimizing deep networks is difficult and current
solvers cannot find the solution. Therefore they propose a deep residual
learning framework. Instead of hoping each few stacked layers directly fit a
desired underlying mapping they explicitly let these layers fit a residual
mapping. The authors hypothesize that letting the stacked layers fit this
residual mapping is easier than letting them directly fit the desired underlying
mapping. \\
Similar to GoogLeNet several of the residual blocks are stacked upon each other
to construct the final network. It ends with a global average pooling layer
followed by fully connected layer with softmax for the classification. 
\begin{figure}
  \centering
  \includegraphics[width=0.4\textwidth]{ResidualBlock}
  \caption{Residual block (taken from \cite{he2016deep}).}
  \label{ResidualBlock}
\end{figure}

\section {Class Activation Mapping}
\label{CAM}
\subsubsection{by Dominique Cheray}
Class Activation Mapping (CAM) is a technique to expose the implicit attention
of Convolutional Neural Networks (CNN) on an image. It highlights the most
informative regions relevant the predicted class \cite{zhou2016learning}. For a
CNN to be used for CAM, its architecture must meet certain requirements. Its
last convolutional layer has to be followed by a layer that performs global
average pooling (GAP) on the convolutional feature maps. The output of the GAP
is then used as features for a fully connected layer that produces the desired
output (categorical or otherwise). \\
The GAP outputs the average of each feature map at the last convolutional layer
and in the fully connected layer a weighted sum of these values is used to
generate the final output. To generate the CAM one now takes the weights of the
fully connected layer belonging to the desired class and the feature maps of the
last convolutional layer and multiplies each feature map by its associated
weight. The weighted feature maps are then added up and upsampled to the size of
the original image. In the resulting image one can identify the image regions
most important to the particular class. Figure \ref{SchematicCAM} shows a
schematic overview of the procedure for generating the class activation maps.
\begin{figure}
  \centering
  \includegraphics[width=0.9\textwidth]{CAM_graphik}
  \caption{Schematic overview of the procedure for generating the class activation maps (taken from
    \cite{zhou2016learning}).}
  \label{SchematicCAM}
\end{figure}

\section{Data Set}
\label{dataset}
\subsubsection{by Dominique Cheray}
The data set we used in our project is a subset of the MPII Human Pose Dataset
\cite{andriluka20142d}. This data set is a state of the art benchmark for
evaluation of articulated human pose estimation. It includes around
25,000 images and covers 410 human activities. The activities cover a wide range
and span from household activities and leisure activities to sports. Each image
of the data set was extracted from a YouTube video and is provided with an
activity label. In addition the preceding and following unannotated frames are
also provided. \\
For the purpose of this project we concentrated on the sports part of the
data set. In order to keep our data set as large as possible and at the same
time the classes balanced, we have selected the sports that contain the most
pictures and a similar number of pictures. The remaining data set consists of
1576 images distributed over 10 classes. These classes are basketball, horseback
riding, martial arts, paddleball, rock climbing, rope skipping, skateboarding,
softball, tennis, golf. Figure \ref{plotgrid} gives a little insight into the
data set. 
\begin{figure}
  \centering
  \includegraphics[width=0.6\textwidth]{plotgrid}
  \caption{Samples of the data set - one row is one class.}
  \label{plotgrid}
\end{figure}

\chapter{Methods}
\section{Classification with the Networks}


\bibliography{literature}

\end{document}
